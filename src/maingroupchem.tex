\begin{lernziele}
Nach diesem Kapitel können Sie für jede Hauptgruppe (1, 2, 13--18) die häufigsten Oxidationsstufen, charakteristische Verbindungen, Strukturmotive, Leittrends und typische Reaktionen benennen und einfache Aufgaben dazu lösen.
\end{lernziele}

\begin{merksatz}
\textbf{Leitmotive:} Metallizität und Polarisierbarkeit nehmen gruppenabwärts zu; Katenation nimmt von C zu Pb ab; \emph{inert pair}-Effekt stabilisiert niedrige Oxidationsstufen bei schweren p-Elementen; diagonale Beziehungen (Li--Mg, Be--Al); Lewis-Acidität der Gruppe-13-Trihalogenide; Hypervalenz via 3c--4e-Bindungen bei schweren p-Elementen.
\end{merksatz}

\section{Gruppe 1 --- Alkalimetalle (Li--Cs)}
\begin{center}
\begin{tabular}{@{}ll@{}}\toprule
Leitoxidationsstufe & \(+1\) (\ce{M+}) \\
Typische Spezies & \ce{M2O}, \ce{M2O2}/\ce{KO2}, \ce{MH}, \ce{M2CO3} \\
Besonderheiten & solv. Elektronen in flüssigem \ce{NH3}; Komplexierung durch Kronenether/Kryptanden \\
\bottomrule\end{tabular}
\end{center}
\textbf{Kernreaktionen/Trends:}
\begin{itemize}
  \item Wasser: \ce{2 M + 2 H2O -> 2 M+ + 2 OH- + H2 (g)} (Heftigkeit Li \(\downarrow\) Cs).
  \item Peroxid/Superoxidbildung: \ce{Na2O2} (Peroxid), \ce{KO2} (Superoxid, paramagnetisch).
  \item In \ce{NH3(l)}: tiefe blaue Lösungen (solvierte Elektronen) \textrightarrow{} starke Reduktionsmittel.
\end{itemize}

\section{Gruppe 2 --- Erdalkalimetalle (Be--Ba)}
\begin{center}
\begin{tabular}{@{}ll@{}}\toprule
Leitoxidationsstufe & \(+2\) (\ce{M^{2+}}); Be oft kovalent, amphoter \\
Typische Spezies & Oxide \ce{MO}, Hydroxide \ce{M(OH)2}, Carbonate \ce{MCO3}, Sulfate \ce{MSO4} \\
Besonderheiten & Diagonalbeziehung Be--Al; \ce{BeCl2} polymer (Brücken-\ce{Cl}); Flammenfärbung Ca/Sr/Ba \\
\bottomrule\end{tabular}
\end{center}
\textbf{Kernreaktionen/Trends:}
\begin{itemize}
  \item Carbonatzersetzung: \ce{MCO3 -> MO + CO2} (leicht für schwere M).
  \item Basizität \ce{M(OH)2} nimmt nach unten zu; Löslichkeit der Sulfate nimmt von \ce{MgSO4} zu \ce{BaSO4} ab.
  \item Wasser: Be kaum, Mg langsam, Ca/Ba rasch: \ce{M + 2 H2O -> M(OH)2 + H2}.
\end{itemize}

\section{Gruppe 13 --- Bor-Gruppe (B, Al, Ga, In, Tl)}
\begin{center}
\begin{tabular}{@{}ll@{}}\toprule
Leitoxidationsstufen & \(+3\); bei Tl auch \(+1\) (inert pair) \\
Typische Motive & Elektronenmangel; starke Lewis-Acidität der Trihalogenide (\ce{BF3}, \ce{AlCl3}) \\
Besonderheiten & \ce{BF3} monomer; \ce{AlCl3} dimer \ce{Al2Cl6} (Brücken-\ce{Cl}); Borane (3c--2e) \\
\bottomrule\end{tabular}
\end{center}
\textbf{Kernreaktionen:}
\begin{itemize}
  \item Addukte: \ce{BF3 + F- -> BF4^-}; \ce{AlCl3 + Cl- -> [AlCl4]^-}.
  \item Hydrolyse: \ce{AlCl3 + 3 H2O -> Al(OH)3 + 3 HCl}.
  \item Hydroborierung (Organik): \ce{BH3·THF + RCH=CH2 -> RCH2-CH2-BH2} \(\rightarrow\) Alkohol nach Oxidation.
\end{itemize}

\section{Gruppe 14 --- Kohlenstoff-Gruppe (C, Si, Ge, Sn, Pb)}
\begin{center}
\begin{tabular}{@{}ll@{}}\toprule
Leitoxidationsstufen & \(+4\) stabil; \(+2\) Stabilität \(\uparrow\) nach unten \\
Typische Motive & Katenation: C \(\gg\) Si \(>\) Ge; Netzwerkbildung (\ce{SiO2}) \\
Besonderheiten & \ce{Sn^{2+}}/\ce{Pb^{2+}} stabil (inert pair); \ce{PbO2} stark oxidierend \\
\bottomrule\end{tabular}
\end{center}
\textbf{Kernreaktionen:}
\begin{itemize}
  \item Hydrolyse: \ce{SiCl4 + 2 H2O -> SiO2(s) + 4 HCl}.
  \item Redox: \ce{SnCl2} als Reduktionsmittel; \ce{PbO2} oxidiert \ce{I-} zu \ce{I2}.
  \item Allotrope: Diamant, Graphit, Silizium-Halbleiter.
\end{itemize}

\section{Gruppe 15 --- Pnictogene (N, P, As, Sb, Bi)}
\begin{center}
\begin{tabular}{@{}ll@{}}\toprule
Leitoxidationsstufen & \(-3, +3, +5\); \(+3\) stabiler bei schweren Elementen \\
Typische Motive & \ce{N2} sehr inert; P-Allotrope (weiß/rot/schwarz) \\
Besonderheiten & Hypervalente Halogenide: \ce{PCl5} (TBP, 3c--4e in Axialpositionen) \\
\bottomrule\end{tabular}
\end{center}
\textbf{Kernreaktionen:}
\begin{itemize}
  \item Basizität: \ce{NH3 + H+ -> NH4+}.
  \item Oxosäuren: \ce{P4O10 + 6 H2O -> 4 H3PO4}.
  \item Halogenide: \ce{PCl5 <=> PCl3 + Cl2}.
\end{itemize}

\section{Gruppe 16 --- Chalkogene (O, S, Se, Te, Po)}
\begin{center}
\begin{tabular}{@{}ll@{}}\toprule
Leitoxidationsstufen & \(-2, +4, +6\); O bevorzugt \(-2\) \\
Typische Motive & Hydride: \ce{H2O} hoher Siedepunkt (H-Brücken) \\
Besonderheiten & S-Allotrope (\ce{S8}); Metallchalkogenide als Halbleiter \\
\bottomrule\end{tabular}
\end{center}
\textbf{Kernreaktionen:}
\begin{itemize}
  \item Kontaktprozess: \ce{SO2 + 1/2 O2 <=> SO3} (Katalysator \ce{V2O5}).
  \item \ce{SO3 + H2SO4 -> H2S2O7 -> H2SO4(aq)}.
  \item Redox: \ce{SO2} (reduzierend) vs. \ce{SO3} (oxidierend).
\end{itemize}

\section{Gruppe 17 --- Halogene (F, Cl, Br, I)}
\begin{center}
\begin{tabular}{@{}ll@{}}\toprule
Leitoxidationsstufen & typ. \(-1\); Cl/Br/I auch \(+1,+3,+5,+7\) \\
Typische Motive & starke Oxidationsmittel; Interhalogene (\ce{ClF3}, \ce{BrF5}, \ce{IF7}) \\
Besonderheiten & Halogenbindung (\(\sigma\)-Loch) in Kristallpackungen \\
\bottomrule\end{tabular}
\end{center}
\textbf{Kernreaktionen:}
\begin{itemize}
  \item Disproportionierung (kalt, verd. Base): \ce{Cl2 + 2 OH- -> Cl- + ClO- + H2O}.
  \item Disproportionierung (heiß, konz. Base): \ce{3 Cl2 + 6 OH- -> 5 Cl- + ClO3- + 3 H2O}.
\end{itemize}

\section{Gruppe 18 --- Edelgase (He, Ne, Ar, Kr, Xe, Rn)}
\begin{center}
\begin{tabular}{@{}ll@{}}\toprule
Reaktivität & sehr gering; Verbindungen v. a. bei Xe, teils Kr \\
Typische Verbindungen & \ce{XeF2}, \ce{XeF4}, \ce{XeF6}, \ce{XeO3}, \ce{XeO4}; \ce{KrF2} \\
Anwendungen & Ar (Schutzgas), Ne (Leuchten), He (Kryo), Xe (Ionenantrieb) \\
\bottomrule\end{tabular}
\end{center}
\textbf{Kernreaktionen:}
\begin{itemize}
  \item Bildung: \ce{Xe + 2 F2 ->[Ni, 200 °C, p] XeF4}.
  \item Hydrolyse: \ce{XeF6 + 3 H2O -> H3XeO6 + 6 HF}.
\end{itemize}

\begin{beispiel}
\textbf{Disproportionierung von Chlor in heißer Lauge.}
\[
\ce{3 Cl2 + 6 OH- -> 5 Cl- + ClO3- + 3 H2O}.
\]
\end{beispiel}

\begin{lernziele}
Sie können nun Gruppentrends (Metallizität, Polarisierbarkeit), die wichtigsten Verbindungsfamilien (Hydride, Halogenide, Oxide/Oxyanionen) und typische Reaktionsmuster (Hydrolyse, Disproportionierung, Redox) erklären und anwenden.
\end{lernziele}
